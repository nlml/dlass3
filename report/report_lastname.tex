\documentclass{article}

% if you need to pass options to natbib, use, e.g.:
% \PassOptionsToPackage{numbers, compress}{natbib}
% before loading nips_2016
%
% to avoid loading the natbib package, add option nonatbib:
% \usepackage[nonatbib]{nips_2016}

\usepackage[final]{nips_2016}

% to compile a camera-ready version, add the [final] option, e.g.:
% \usepackage[final]{nips_2016}

\usepackage[utf8]{inputenc} % allow utf-8 input
\usepackage[T1]{fontenc}    % use 8-bit T1 fonts
\usepackage{hyperref}       % hyperlinks
\usepackage{url}            % simple URL typesetting
\usepackage{booktabs}       % professional-quality tables
\usepackage{amsfonts}       % blackboard math symbols
\usepackage{nicefrac}       % compact symbols for 1/2, etc.
\usepackage{microtype}      % microtypography

\title{Report for the Deep Learning Course Assignment 2 }

% The \author macro works with any number of authors. There are two
% commands used to separate the names and addresses of multiple
% authors: \And and \AND.
%
% Using \And between authors leaves it to LaTeX to determine where to
% break the lines. Using \AND forces a line break at that point. So,
% if LaTeX puts 3 of 4 authors names on the first line, and the last
% on the second line, try using \AND instead of \And before the third
% author name.

\author{
  David S.~Hippocampus \\
  \texttt{hippo@cs.cranberry-lemon.edu}
}

\begin{document}
% \nipsfinalcopy is no longer used

\maketitle

\begin{abstract}
Should contain information about the current task and the summary of the study of the CNN models on CIFAR10 dataset.

\end{abstract}

\section{Task 1}

Use feature\_extraction() in order to compute features for test samples at layer fc2. Use these features to visualize learned space by the help of t-SNE. Include the visualization in your report.

Note: You can download and use the implementation for t-SNE from here. You can also use sklearn implementation of t-SNE.

Train 10 linear one-vs-rest classifiers for each class and report the performances. Can you draw similar conclusions by just looking at the visualization and thinking about separability and how good classes are represented by the model?

Note: You can use any linear classifier implementation available as long as you preserve consistency in experiments. See scikitlearn, libsvm, liblinear, VLFeat and etc.

Repeat the above experiments for flatten and fc1. Discuss the results. What is more suited for classification purposes: flatten, fc1 or fc2?

Can you improve test performance by using regularization techniques? Try L2 weight regularization on the fully-connected layers. You can also try others regularization techniques like dropout and batch normalization. Report your conclusions in your report with experimental support.



\section{Task 2}
Should contain all needed information about Task 2 and report of all your experiments for that task.

\section{Task 3}
Should contain all needed information about Task 3 and report of all your experiments for that task.

\section{Conclusion}
Should contain conclusion of this study.

\section*{References}

\small

[1] Alexander, J.A.\ \& Mozer, M.C.\ (1995) Template-based algorithms
for connectionist rule extraction. In G.\ Tesauro, D.S.\ Touretzky and
T.K.\ Leen (eds.), {\it Advances in Neural Information Processing
  Systems 7}, pp.\ 609--616. Cambridge, MA: MIT Press.

[2] Bower, J.M.\ \& Beeman, D.\ (1995) {\it The Book of GENESIS:
  Exploring Realistic Neural Models with the GEneral NEural SImulation
  System.}  New York: TELOS/Springer--Verlag.

[3] Hasselmo, M.E., Schnell, E.\ \& Barkai, E.\ (1995) Dynamics of
learning and recall at excitatory recurrent synapses and cholinergic
modulation in rat hippocampal region CA3. {\it Journal of
  Neuroscience} {\bf 15}(7):5249-5262.

\end{document}
